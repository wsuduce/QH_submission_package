\documentclass{JHEP}
\usepackage{microtype}
\usepackage{graphicx}
\usepackage{amsmath,amssymb}
\usepackage{bm}
\usepackage[hidelinks]{hyperref}
\usepackage[capitalise]{cleveref}

% JCAP formatting preferences
\def\be{\begin{equation}}
\def\ee{\end{equation}}

\begin{document}

\title{Evidence for Universal Scale Coupling Across 61 Orders of Magnitude}

\author[a]{Adam Murphy}
\emailAdd{adam@impactme.ai}

\affiliation[a]{Independent Researcher}

\abstract{%
We present evidence for a universal scale-coupling constant $\delta = 0.502 \pm 0.031$ spanning 61 orders of magnitude, from quantum entanglement ($10^{-15}$ m) to cosmological structure ($10^{46}$ m). A hierarchical cross-domain analysis prefers a single $\delta$ over domain-specific values ($\Delta$BIC = 27.4). Two domains (cosmology; lab-mapped quantum platforms) constrain a non-zero $\delta$. Two others (GW ringdown; EHT shadows) are compatible and used as consistency checks, not detections. A laboratory-measured ratio $\beta/\alpha = 0.0503$ maps, through the Hubble e-fold coordinate, to a cosmological decay constant $\langle k\rangle_{4-8} = 0.530$ that matches JWST/MIDIS ($0.523 \pm 0.058$) without tuned parameters.

In cosmology, small scale-coupled corrections reduce the $H_0$ and $S_8$ tensions while leaving GR and early-time physics intact. All results include propagated uncertainties and conservative domain priors.

We commit to concrete, near-term tests (central values with propagated theory error): LIGO/Virgo/KAGRA O4--O5: ringdown overtone scaling $f \approx 420$ Hz $\times$ $(80 M_\odot/M_f)$ for 70--90 $M_\odot$ remnants with $a^* \lesssim 0.7$; Euclid ($z \approx 1$): BAO distance indicator shift $\approx +0.22\%$ ($\approx +0.33$ Mpc relative to a 147.0 Mpc fiducial); DESI ($z = 0.5$): dark-energy state $w \approx -1.009$.

Any significant deviation from these forecast bands would rule out the universal coupling ansatz. Collectively, these results indicate that a single parameter ($\delta$) organizes small residuals across domains. Quantum Harmonia offers one interpretation; the parameters stand on their own and warrant explanation.
}

\keywords{dark energy, modified gravity, quantum mechanics, cosmological tensions}

\maketitle

% Import the main content
\input{content.tex}

% References
\bibliographystyle{JHEP}
\bibliography{refs}

\end{document}